%!TEX TS-program = xelatex
\documentclass[]{friggeri-cv}
\addbibresource{bibliography.bib}
\usepackage[none]{hyphenat}

\begin{document}
\header{Michelle}{Fullwood}


% In the aside, each new line forces a line break
\begin{aside}
  \section{about}
	Cambridge, MA
    \href{mailto:maf@mit.edu}{maf@mit.edu}
    \href{http://www.mit.edu/~maf}{www.mit.edu/\textasciitilde maf}
    \href{http://github.com/michelleful}{github.com/michelleful}
  \section{programming}
    Python, Django, Flask
    JavaScript, jQuery
	R, Perl, Matlab/Octave
	SQL, git
  \section{languages}
	{\footnotesize\addfontfeature{Color=lightgray}native} English
	{\footnotesize\addfontfeature{Color=lightgray}advanced} French
	         Mandarin
	{\footnotesize\addfontfeature{Color=lightgray}intermediate} Arabic
	{\footnotesize\addfontfeature{Color=lightgray}elementary} Japanese
	           Hungarian
%	           Latin
  \section{awards}
    {\footnotesize\addfontfeature{Color=lightgray}National Science Foundation}
    {\footnotesize\addfontfeature{Color=lightgray}2011} NSF Graduate Research Fellowship
    {\footnotesize\addfontfeature{Color=lightgray}Cornell University} 
    {\footnotesize\addfontfeature{Color=lightgray}2004} Merrill Scholar
    {\footnotesize\addfontfeature{Color=lightgray}2001} College Scholar
    {\footnotesize\addfontfeature{Color=lightgray}2000} Dean's Scholar
    {\footnotesize\addfontfeature{Color=lightgray}2002,2004} Achievement Awards in Arabic
    {\footnotesize\addfontfeature{Color=lightgray}2000--2004} Dean's List
    {\footnotesize\addfontfeature{Color=lightgray}Government of Singapore}
    {\footnotesize\addfontfeature{Color=lightgray}2000} PSC Overseas Merit Scholarship 
\end{aside}

\section{summary}

Graduate student in linguistics with a background in speech and language processing and web development in Python
% TODO

\section{education}

\begin{entrylist}
  \entry
    {since 2010}
    {Ph.D. program in Linguistics, Massachusetts Institute of Technology}
    {}
    {%National Science Foundation Graduate Research Fellow \\
     Research on problems in computational morphology and formal phonology \\
     Coursework in linguistics, machine learning and Bayesian techniques   \\
     Teaching assistant in introductory linguistics and phonology
    }
  \entry
    {2000--2004}
    {B.A. in Linguistics and Mathematics, Cornell University}
    {}
    {Graduated \emph{magna cum laude} and with distinction in all subjects, 4.0 GPA \\
	Study abroad, Fall 2003: Budapest Semesters in Mathematics}
\end{entrylist}

\section{experience}

\begin{entrylist}
  \entry
    {2008--2010}
    {Web Developer, Imperial Consulting}
    {Boston, MA}
    {Developed web applications, both front and back end, in Python, Django and jQuery,
     for clients in fields ranging from education to biomedical to finance}
  \entry
    {2008--2009}
    {External Consultant, Centre for Strategic Infocomm Technologies}
    {}
    {Advised client on evaluation procedures in speech and natural language processing}
  \entry
    {2004--2008}
    {R\&D Engineer, later Consultant, \\ Centre for Strategic Infocomm Technologies}
    {Singapore}
    {Researched techniques and built engines for speech recognition, language and speaker identification, and cross-language information retrieval \\
    Managed projects to evaluate and acquire systems}
  \entry
    {2004}
    {Summer Intern, PARC}
    {Palo Alto, CA}
    {Engineered a computational Lexical Functional Grammar to cover the basic sentence constructions of Modern Standard Arabic \\
Built a root-and-pattern-based finite state Arabic morphological analyzer}
\end{entrylist}

\section{projects}

\begin{entrylist}
  \entry
    {since 2012}
    {Bayesian inference of non-concatenative morphology}
    {}
    {Extending state-of-the-art Bayesian techniques for morphological learning to the more complex case of non-concatenative morphology (Arabic, Hebrew) \\
    Wrote a Markov Chain Monte Carlo (MCMC) sampler in Python to perform inference on a Bayesian generative model of non-concatenative morphology}
  \entry
    {2014}
    {Fuzzy Arabic Dictionary}
    {\href{http://fuzzyarabic.herokuapp.com}{fuzzyarabic.herokuapp.com}}
    {Developed a beginner-friendly Arabic dictionary that can be queried without knowing how to spell in Arabic, mashing up the Yamli transliteration service and the free Buckwalter Arabic Morphological Analyzer}
  \entry
    {since 2013}
    {Volunteer tutor, PyLadies Boston}
    {}
    {Deliver presentations on beginner and intermediate Python topics and tutor women who are learning to program in Python}
\end{entrylist}

\end{document}
